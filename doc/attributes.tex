% !TEX root = SoLoud.tex
%===============================================================================

\chapter{Core: Attributes}

\section{Soloud.getVolume() / Soloud.setVolume()}

These functions can be used to get and set a sound's current volume setting.

\begin{lstlisting}{frame=single, framerule=1pt}
float v = soloud.getVolume(h); // Get current volume
soloud.setVolume(h, v * 2);    // Double it
\end{lstlisting}

Note that the volume is the "volume setting", and the actual volume will depend on the sound source. Namely, a whisper will most likely be more quiet than a scream, even if both are played at the same volume setting.

If an invalid handle is given to getVolume, it will return 0.

\section{Soloud.getPan() / Soloud.setPan()}

These functions can be used to get and set a sound's current pan setting.

\begin{lstlisting}{frame=single, framerule=1pt}
float v = soloud.getPan(h); // Get current pan
soloud.setPan(h, v - 0.1);  // Little bit to the left
\end{lstlisting}

The range of the pan values is -1 to 1, where -1 is left, 0 is middle and and 1 is right. Setting value outside this range may cause undefined behavior.

SoLoud calculates the left/right volumes from the pan to keep a constant volume; to set the volumes directly, use setPanAbsolute.

If an invalid handle is given to getPan, it will return 0.

\section{Soloud.setPanAbsolute()}

These function can be used to set the left/right volumes directly.

\begin{lstlisting}{frame=single, framerule=1pt}
soloud.setPanAbsolute(h, 1, 1); // full blast
\end{lstlisting}

Note that this does not affect the value returned by getPan.

If an invalid handle is given to getPan, it will return 0.

\section{Soloud.getSamplerate() / Soloud.setSamplerate()}

These functions can be used to get and set a sound's base sample rate.

\begin{lstlisting}{frame=single, framerule=1pt}
float v = soloud.getSamplerate(h); // Get the base sample rate
soloud.setSamplerate(h, v * 2);    // Double it
\end{lstlisting}

Setting the value to 0 will cause undefined behavior, likely a crash.

To adjust the play speed, while leaving the base sample rate alone, 
use setRelativePlaySpeed instead.

If an invalid handle is given to getSamplerate, it will return 0.

\section{Soloud.getRelativePlaySpeed() / Soloud.setRelativePlaySpeed()}

These functions can be used to get and set a sound's relative play speed.

\begin{lstlisting}{frame=single, framerule=1pt}
float v = soloud.getRelativePlaySpeed(h); // Get relative play speed
soloud.setRelativePlaySpeed(h, v * 0.5f); // Halve it
\end{lstlisting}

Setting the value to 0 will cause undefined behavior, likely a crash.

Change the relative play speed of a sample. This changes the effective sample rate while leaving the base sample rate alone.

Note that playing a sound at a higher sample rate will require SoLoud to request more samples from the sound source, which will require more memory and more processing power. Playing at a slower sample rate is cheaper.

If an invalid handle is given to getRelativePlaySpeed, it will return 1.

\section{Soloud.getProtectChannel() / Soloud.setProtectChannel()}

These functions can be used to get and set a sound's protection state.

\begin{lstlisting}{frame=single, framerule=1pt}
int v = soloud.getProtectChannel(h);   // Get the protection state
if (v) soloud.setProtectChannel(h, 0); // Disable if protected
\end{lstlisting}

Normally, if you try to play more sounds than there are channels, SoLoud will kill off the oldest playing sound to make room. This will most likely be your background music. This can be worked around by protecting the sound.

If all sounds are protected, the result will be undefined.

If an invalid handle is given to getProtectChannel, it will return 0.

\section{Soloud.getPause() / Soloud.setPause()}

The setPause function can be used to pause, or unpause, a sound.

\begin{lstlisting}{frame=single, framerule=1pt}
if (soloud.getPause(h)) hum_silently();
soloud.setPause(h, 0); // resumes playback
\end{lstlisting}

Note that even if a sound is paused, its channel may be taken over. Trying to resume a sound that's no longer in a channel doesn't do anything.

If the handle is invalid, the getPause will return 0.

\section{Soloud.setPauseAll()}

The setPauseAll function can be used to pause, or unpause, all sounds.

\begin{lstlisting}{frame=single, framerule=1pt}
soloud.setPauseAll(h, 0); // resumes playback of all channels
\end{lstlisting}

Note that this function will overwrite the pause state of all channels at once. If your game uses this to pause/unpause the sound while the game is paused, do note that it will also pause/unpause any sounds that you may have paused/unpaused separately.

\section{Soloud.setFilterParameter()}

Sets a parameter for a live instance of a filter. The filter must support changing of live parameters; otherwise this call does nothing.

\begin{lstlisting}{frame=single, framerule=1pt}
soloud.setFilterParameter(bar,lp,cutoff,1000); 
// set bar's low pass to cut at 1000hz
\end{lstlisting}

\section{Soloud.getFilterParameter()}
Gets a parameter from a live instance of a filter. The filter must support changing of live parameters; otherwise this call returns zero.
\begin{lstlisting}{frame=single, framerule=1pt}
float v = soloud.getFilterParameter(bar,lp,cutoff); 
// get bar's low pass cutoff
\end{lstlisting}

