% !TEX root = SoLoud.tex
%===============================================================================

\chapter{SoLoud::AudioSource}

All audio sources share some common functions. Some of the functionality depends on the audio source itself; it may be that some parameter does not make sense for a certain audio source, or it may be that it has not been implemented for other reasons.

For example, if you stream a live radio station, looping does not make much sense.

\section{AudioSource.setLooping()}

This function can be used to set a sample to play on repeat, instead of just playing once.

\begin{lstlisting}{frame=single, framerule=1pt}
amenbreak.setLooping(1); // let the beat play on
\end{lstlisting}

Note that some audio sources may not implement this behavior.

\section{AudioSource.setFilter()}

This function can be used to set or clear the filters that should be applied to the sounds generated via this audio source.

\begin{lstlisting}{frame=single, framerule=1pt}
speech.setFilter(0, blackmailer); // Disguise the speech
\end{lstlisting}

Setting the filter to NULL will clear the filter. This will not affect already playing sounds.
By default, up to four filters can be applied. This value can be changed through a constant in the soloud.h file.

\section{AudioSource.setSingleInstance()}

This function can be used to tell SoLoud that only one instance of this sound may be played at the same time.

\begin{lstlisting}{frame=single, framerule=1pt}
menuselect.setSingleInstance(1); // Only play it once, sam
\end{lstlisting}
