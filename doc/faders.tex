% !TEX root = SoLoud.tex
%===============================================================================

\chapter{Core: Faders}

\section{Overview}

Faders are a convenient way of performing some common audio tasks without having to add complex code into your application.

The most common use for the faders is to fade audio in or out, adding nice touches and polish.

Let's say you're exiting a bar and entering the street.

\begin{lstlisting}{frame=single, framerule=1pt}
soloud.fadeVolume(bar_ambience, 0, 2); // fade bar out in 2 seconds
soloud.scheduleStop(bar_ambience, 2);  // stop the bar ambience after fadeout
street_ambience = soloud.play(cars, 0);// start steet ambience at 0 volume
soloud.setProtectChannel(street_ambience, 1); // protect it
soloud.fadeVolume(street_ambience, 1, 1.5f); // fade street in in 1.5
\end{lstlisting}

Or let's say you're quiting your game.

\begin{lstlisting}{frame=single, framerule=1pt}
soloud.fadeGlobalVolume(0, 1); // Fade out global volume in 1 second
\end{lstlisting}

The faders are only evaluated once per mix function call - in other words, whenever the back end requests samples from SoLoud, which is likely to be in chunks of 20-100ms, which is smoothly enough for most uses.

The exception is volume (which includes panning), which gets interpolated on per-sample basis to avoid artifacts.

The starting value for most faders is the current value.

\section{Soloud.fadeVolume()}

Smoothly change a channel's volume over specified time.

\begin{lstlisting}{frame=single, framerule=1pt}
soloud.fadeVolume(orchestra, 1, 60); // The orchestra creeps in for a minute
\end{lstlisting}

The fader is disabled if you change the channel's volume with setVolume()

\section{Soloud.fadePan()}

Smoothly change a channel's pan setting over specified time.

\begin{lstlisting}{frame=single, framerule=1pt}
soloud.setPan(racecar, -1); // set start value
soloud.fadePan(racecar, 1, 0.5); // Swoosh!
\end{lstlisting}

The fader is disabled if you change the channel's panning with setPan() or setPanAbsolute()

\section{Soloud.fadeRelativePlaySpeed()}

Smoothly change a channel's relative play speed over specified time.

\begin{lstlisting}{frame=single, framerule=1pt}
soloud.fadeRelativePlaySpeed(hal, 0.1, 6); // Hal's message slows down
\end{lstlisting}

The fader is disabled if you change the channel's play speed with setRelativePlaySpeed()

\section{Soloud.fadeGlobalVolume()}

Smoothly change the global volume over specified time.

\begin{lstlisting}{frame=single, framerule=1pt}
soloud.fadeGlobalVolume(0, 2); // Fade everything out in 2 seconds
\end{lstlisting}

The fader is disabled if you change the global volume with setGlobalVolume()

\section{Soloud.schedulePause()}

After specified time, pause the channel

\begin{lstlisting}{frame=single, framerule=1pt}
soloud.fadeVolume(jukebox, 0, 2); // Fade out the music in 2 seconds
soloud.schedulePause(jukebox, 2);    // Pause the music after 2 seconds
\end{lstlisting}

The scheduler is disabled if you set the pause state with setPause() or setPauseAll().

\section{Soloud.scheduleStop()}

After specified time, stop the channel

\begin{lstlisting}{frame=single, framerule=1pt}
soloud.fadeVolume(applause, 0, 10); // Fade out the cheers for 10 seconds
soloud.scheduleStop(applause, 10);     // Stop the sound after 10 seconds
\end{lstlisting}

There's no way (currently) to disable this scheduler.

\section{Soloud.oscillateVolume()}

Set fader to oscillate the volume at specified frequency.

\begin{lstlisting}{frame=single, framerule=1pt}
soloud.oscillateVolume(murmur, 0, 0.2, 5); // murmur comes and goes
\end{lstlisting}
The fader is disabled if you change the channel's volume with setVolume()

\section{Soloud.oscillatePan()}

Set fader to oscillate the panning at specified frequency.

\begin{lstlisting}{frame=single, framerule=1pt}
soloud.oscillatePan(ambulance, -1, 1, 10); // Round and round it goes
\end{lstlisting}
The fader is disabled if you change the channel's panning with setPan() or setPanAbsolute()

\section{Soloud.oscillateRelativePlaySpeed()}
Set fader to oscillate the relative play speed at specified frequency.

\begin{lstlisting}{frame=single, framerule=1pt}
soloud.oscillateRelativePlaySpeed(vinyl, 0.9, 1.1, 3); // Wobbly record
\end{lstlisting}

The fader is disabled if you change the channel's play speed with setRelativePlaySpeed()

\section{Soloud.oscillateGlobalVolume()}
Set fader to oscillate the global volume at specified frequency.

\begin{lstlisting}{frame=single, framerule=1pt}
soloud.oscillateGlobalVolume(0.5, 1.0, 0.2); // Go crazy
\end{lstlisting}

The fader is disabled if you change the global volume with setGlobalVolume()

\section{Soloud.fadeFilterParameter()}

Fades a parameter on a live instance of a filter. The filter must support changing of live parameters; otherwise this call does nothing.

\begin{lstlisting}{frame=single, framerule=1pt}
soloud.fadeFilterParameter(bar,lp,cutoff,1000,1); 
// Fades bar's low pass to cut at 1000hz
\end{lstlisting}

\section{Soloud.oscillateFilterParameter()}
Oscillates a parameter on a live instance of a filter. The filter must support changing of live parameters; otherwise this call does nothing.

\begin{lstlisting}{frame=single, framerule=1pt}
soloud.setFilterParameter(bar,lp,cutoff,500,1000,2); 
// Oscillates the bar's lp filter's cutoff between 500 and 1kHz
\end{lstlisting}

