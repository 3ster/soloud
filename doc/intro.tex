% !TEX root = SoLoud.tex
%===============================================================================

\chapter{Introduction}

SoLoud is an easy to use, free, portable c/c++ audio engine for games.

\section {How easy?}
The engine has been designed to make simple things easy, while not making harder things impossible. Here's a code snippet that initializes the library, loads a sample and plays it:

\begin{lstlisting}{frame=single, framerule=1pt}
// Declare some variables
SoLoud::Soloud soloud; // Engine core
SoLoud::Wav sample;    // One sample

// Initialize SoLoud (with SDL)
SoLoud::sdl_init(&soloud);

sample.load("pew_pew.wav"); // Load a wave file
soloud.play(sample);        // Play it
\end{lstlisting}

The primary form of use the interface is designed for is "fire and forget" audio. In many games, most of the time you don't need to modify a sound's parameters on the fly - you just find an event, like an explosion, and trigger a sound effect. SoLoud handles the rest.

If you need to alter some aspect of the sound after the fact, the "play" function returns a handle you can use. For example:

\begin{lstlisting}{frame=single, framerule=0pt}
int handle = soloud.play(sample);         // Play the sound
soloud.setVolume(handle, 0.5f);           // Set volume; 1.0f is "normal"
soloud.setPan(handle, -0.2f);             // Set pan; -1 is left, 1 is right
soloud.setRelativePlaySpeed(handle, 0.9f);// Play a bit slower; 1.0f is normal
\end{lstlisting}

If the sound doesn't exist anymore (either it's ended or you've played so many sounds at once it's channel has been taken over by some other sound), the handle is still safe to use - it just doesn't do anything.

\section{How free?}
SoLoud is released under the ZLib/LibPNG license. That means you can use it in free or commercial applications as much as you want. You can modify it. You don't need to give the changes back. You don't need to release the source code. You don't need to add a splash screen. You don't need to mention it in your printed manual.

Basically the only things the license forbids are suing me, or claiming that you made SoLoud. If you redistribute the source code, the license needs to be there. But not with the binaries.

Parts of the SoLoud package were not made by me, and those either have a similar license, or more permissive (such as Unlicense, CC0, WTFPL or Public Domain).

\section{There's a Catch, Right?}

SoLoud quite probably doesn't have all the features you'd find in a commercial library like FMOD or WWISE. There's no artist tools or engine integration. There's no 3d audio. Output is, currently, limited to stereo.

It quite probably isn't as fast. As of this writing, it has no specialized assembler optimizations, for any platform.

It certainly doesn't come with the support you get from a commercial library.

If you're planning to make a multi-million budgeted console game, this library is (probably) not for you. Feel free to try it though =)
