% !TEX root = SoLoud.tex
%===============================================================================

\chapter{Legal}

SoLoud, like everything else, stands on the shoulders of giants; however, care has been taken to only incorporate source code that is under liberal licenses, namely ZLib/LibPNG, CC0 or public domain, or similar, like WTFPL or Unlicense, where you don't need to include mention of the code in your documentation or splash screens or any such nonsense.

\section{SoLoud Proper}

SoLoud proper is licensed under the ZLib/LibPNG license. The code is a clean-room implementation with no outside sources used.

\scriptsize
\begin{verbatim}
SoLoud audio engine
Copyright (c) 2013 Jari Komppa

This software is provided 'as-is', without any express or implied
warranty. In no event will the authors be held liable for any damages
arising from the use of this software.

Permission is granted to anyone to use this software for any purpose,
including commercial applications, and to alter it and redistribute it
freely, subject to the following restrictions:

   1. The origin of this software must not be misrepresented; you must not
   claim that you wrote the original software. If you use this software
   in a product, an acknowledgment in the product documentation would be
   appreciated but is not required.

   2. Altered source versions must be plainly marked as such, and must 
   not be misrepresented as being the original software.

   3. This notice may not be removed or altered from any source
   distribution.
\end{verbatim}
\normalsize

\section{OGG Support}
The OGG support in the Wav and WavStream sound sources is based on stb\_vorbis by Sean Barrett, and it's in the public domain. You can find more information (and latest version) at http://nothings.org/stb\_vorbis/

\section{Speech Synthesizer}
The speech synth is based on rsynth by the late Nick Ing-Simmons (et al). He described the legal status as:

\scriptsize
\begin{verbatim}
    This is a text to speech system produced by
    integrating various pieces of code and tables
    of data, which are all (I believe) in the
    public domain.
\end{verbatim}
\normalsize

Since then, the rsynth source code has passed legal checks by several open source organizations, so it "should" be pretty safe.

The primary copyright claims seem to have to do with text-to-speech dictionary use, which I've removed completely.

I've done some serious refactoring, clean-up and feature removal on the source, as all I need is "a" free, simple speech synth, not a "good" speech synth. Since I've removed a bunch of stuff, this is probably safer public domain release than the original.

I'm placing my changes in public domain as well, or if that's not acceptable for you, then CC0: http://creativecommons.org/publicdomain/zero/1.0/.

The SoLoud interface files (soloud\_speech.*) are under the same ZLib/LibPNG license as the other SoLoud bits.

\section{Fast Fourier Transform (FFT)}

FFT calculation is provided by a fairly simpple implementation by Stephan M. Bernsee, under the Wide Open License:

\scriptsize
\begin{verbatim}
COPYRIGHT 1996 Stephan M. Bernsee <smb [AT] dspdimension [DOT] com>

						The Wide Open License (WOL)

Permission to use, copy, modify, distribute and sell this software and its
documentation for any purpose is hereby granted without fee, provided that
the above copyright notice and this license appear in all source copies. 
THIS SOFTWARE IS PROVIDED "AS IS" WITHOUT EXPRESS OR IMPLIED WARRANTY OF
ANY KIND. See http://www.dspguru.com/wol.htm for more information.
\end{verbatim}
\normalsize
