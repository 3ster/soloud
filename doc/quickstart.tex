% !TEX root = SoLoud.tex
%===============================================================================

\chapter{Quick Start}

Note: SoLoud supports several different "back ends", but this quickstart guide assumes you're using SDL; the primary difference is that to use a different back-end, you call the desired back-end's init function (instead of SDL's).

\section{Download SoLoud}

First, you need to download SoLoud sources. You can find the downloads on the http://soloud-audio.com/download.html page.

\section{Add files to your project}

You can go the lazy way and just add all of the sources to your project, or you can copy the things you need to a single directory and include those. You'll need the core files, and quite likely the wav files. If you need the speech synth, include those, too.

\section{Include files}

In order to use a certain feature of SoLoud, you need to include its include file. You might have, for instance:

\begin{lstlisting}{frame=single, framerule=1pt}
#include "soloud.h"
#include "soloud_wav.h"
\end{lstlisting}

\section{Variables}

You need at least the SoLoud engine core, and one or more of the sound source variables. If you're using five different sound effect wav files, you need five SoLoud::Wav objects. You can play one object any number of times, even on top of each other.

Where to place these is up to you. Globals work, as do allocation from heap, including in a class as members, etc.

\begin{lstlisting}{frame=single, framerule=1pt}
SoLoud::Soloud gSoloud;
SoLoud::Wav gWave;
\end{lstlisting}

\section{Initialize SoLoud}

After your SDL\_Init call, include a call to initialize SoLoud. The call takes a pointer to the SoLoud core object.

\begin{lstlisting}{frame=single, framerule=1pt}
SoLoud::sdl_init(&gSoloud);
\end{lstlisting}

\section{Set up sound sources}

This step varies from one sound source to another, but basically you'll load your wave files here.

\begin{lstlisting}{frame=single, framerule=1pt}
gWave.load("pew_pew.wav");
\end{lstlisting}

\section{Play sounds}

Now you're ready to play the sounds. Place playing commands wherever you need sound to be played.

\begin{lstlisting}{frame=single, framerule=1pt}
gSoloud.play(gWave);
\end{lstlisting}

Note that you can play the same sound several times, and it doesn't cut itself off.

\section{Take control of the sound}

You can adjust various things about the sound you're playing if you take the handle.

\begin{lstlisting}{frame=single, framerule=1pt}
int x = gSoloud.play(gWave);
gSoloud.setPan(x, -0.2f);
\end{lstlisting}

Read the soloud.h header file (or this documentation) for further things you can do.

\section{Cleanup}

After you've done, remember to clean up. If you don't, the audio thread may do stupid things while the application is shutting down.

\begin{lstlisting}{frame=single, framerule=1pt}
gSoloud.deinit();
\end{lstlisting}

\section{Enjoy}

And you're done!